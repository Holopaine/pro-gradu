\documentclass[utf8]{gradu3}
% Jos työ on kandidaatintutkielma eikä pro gradu, käytä ylläolevan asemesta
%\documentclass[utf8,bachelor]{gradu3} Jos kirjoitat englanniksi, käytä
%ylläolevan asemesta \documentclass[utf8,english]{gradu3} tai
%\documentclass[utf8,bachelor,english]{gradu3}

\usepackage{graphicx} % kuvien mukaan ottamista varten

\usepackage{amsmath} % hyödyllinen jos tekstisi sisältää matikkaa,
                     % ei pakollinen

\usepackage{booktabs} % hyvä kauniiden taulukoiden tekemiseen

% HUOM! Tämän tulee olla viimeinen \usepackage koko dokumentissa!
\usepackage[bookmarksopen,bookmarksnumbered,linktocpage]{hyperref}

\addbibresource{pro-gradu.bib} % Lähdetietokannan tiedostonimi

\begin{document}

\title{Tutkimussuunnitelma}
%\translatedtitle{Usage of the {gradu3} document class for \LaTeX\ theses}
%\studyline{Kaikki opintosuunnat} \avainsanat{} \keywords{} \tiivistelma{}
%\abstract{}

\author{Lauri Holopainen}
\contactinformation{\texttt{lauri.m.s.holopainen@jyu.fi}}
% jos useita tekijöitä, anna useampi \author-komento \supervisor{Ohjaamaton työ}
% jos useita ohjaajia, anna useampi \supervisor-komento

% \type{tutkielmapohjan esimerkki ja käsikirja} % et tarvitse tätä riviä
% tutkielmassa!

\maketitle

\mainmatter

\chapter{Johdanto}
% - Bugit
%   * Käytännön tärkeys.
%   * Tieteellinen merkitys.
Ohjelmistokehityksen ja  ja ohjelmistojen ylläpidon keskeisiä ongelmia ovat
ohjelmointivirheet eli bugit. Mitä myöhemmässä vaiheessa bugit huomataan, sitä
enemmän resursseja, siis työaikaa, voi olettaa kuluvan niiden korjaamiseen. Olen
kiinnostunut siitä, kuinka bugikorjauksiin kuluvaa aikaa voisi arvioida
matemaattisten mallien avulla. Tutkielmassani aion selvittää, voisiko nyt jo
kehitettyjä malleja muuttaa tarkemmiksi tuomalla samaan malliin parhaita
piirteitä useista tutkimuksista.

Modernissa ketterässä ohjelmistokehitystyössä hyödynnetään usein erilaisia
tikettijärjestelmiä bugien ja kehitysideoiden raportointiin. Tiketit voivat
saada alkunsa niin loppukäyttäjien, kuin kehittäjäorganisaation huomioista. Kun
tiketin vastaanottaja kuittaa tehtävän työn alle, hän arvioi tarvittavan
ohjelmistomuutoksen kriittisyyden, nimittää sen jollekin kehittäjälle, ja arvioi
siihen kuluvan ajan. Kuluvan ajan sijaan voidaan myös arvioida tehtävän
haasteellisuutta (\textit{story points}). Onnistunut tehtävän vaatiman ajan
arviointi voi auttaa kehitysprojektin vetäjiä suunnittelemaan aikataulun
suunnittelussa ja resurssien suuntaamisessa. Loppukäyttäjiä hyvät
korjausaika-arviot voivat myös hyödyttää esimerkiksi oman työnsä suunnittelussa
tai sen arvioinnissa, että onko järkevää jäädä odottamaan korjauksen
valmistumista.

Bugikorjausten valmistumisaikaa on yritetty arvioida louhimalla tietoa
tikettijärjestelmiä, ja luomalla tilastollisia malleja muun muassa koneoppimisen
avulla. Yksi käytetyimpiä mittareita on aika tiketin avaamisesta sen
sulkeutumiseen. Tutkimuksissa saadut tulokset ovat kuitenkin olleet välillä
ristiriitaisia, eivätkä aina kovinkaan tarkkoja. Vaikuttaisi siis siltä, että
lisätutkimuselle on tarvetta mikäli näitä malleja halutaan soveltaa käytännössä
osana.

\chapter{Kirjallisuuskartoitus}
% - menetelmä: voidaan esitellä hakusanat, hakuprosessi, hakukoneet ja
%   tietokannat (hyödyllistä tietoa kirjata itselle muistiin, ei välttämättä
%   tule lopulliseen graduun)
% - tulokset (tiivistetty kuvaus löytyneistä artikkeleista)
% - kerrotaan, mitä tutkittavasta aiheesta tiedetään entuudestaan metodilähteitä
%   mainittava0

% - Tiedonlouhinta
% - Mining software repositories Issue resolution time/bug fixing time
%   estimation.

Tiedonlouhinnan voi määritellä oleellisen tiedon ja kiinnostavien kaavojen
löytämiseksi suuresta datamassasta \parencite[][8]{han-data_mining}. Alalla
käytetään monenlaisia tilastotieteen, matematiikan ja tietojenkäsittelyn
tekniikoita \parencite{clifton-2019}. 

Ohjelmistokehityksessä hyödynnetään usein hajautettuja
versiohallintajärjestelmiä kuten Git, sekä internet-palveluja kuten Github ja
Bitbucket tietovarastojen (\textit{repository}) hallintaan. Nämä palvelut
tarjoavat usein lähdekoodin jakamisen lisäksi myös muita ominaisuuksia, kuten
jatkuvan integraation kanavan, pull requestit, ja bugiraportit. Mining software
repositories (MSR) on tiedonlouhinnan osa-alue, joka on erikoistunut
ohjelmistokehityksessä syntyvän strukturoidun ja tekstimuotoisen datan
käsittelyyn. Tutkimuksen kohteita ovat esimerkiksi koodin mätäneminen, muutosten
riippuvuus, bugien ennakointi ja syyt, sekä kehittäjien yhteistyö.
\parencite{guemes-pena-emerging_topics}

\section{Bugien korjausajan arviointi}
Malleja ohjelmistobugien korjausajan arviointiin on kehitetty useissa
tutkimuksissa 2010-luvulla. Dataa on kerätty niin avoimen kuin suljetun
lähdekoodin projekteista. Näistä valtaosa on julkaistu konfferenssipapereina.
Joistakin löytyy myös pidempi artikkelimuotoinen teksti. Kirjallisuutta on
löytynyt eniten ACM DL ja IEEE -tietokannoista, sekä seuraamalla viitteitä
snowballing-tekniikalla. Tutkimukset eroavat muun muassa mallin kehityksessä
käytettyjen projektien määrässä, hyödynnetyissä attribuuteissa ja
koneoppimisalgoritmeissa.

\textcite{Giger-2010} tutkivat bugien luokittelua hitaasti ja nopeasti
ratkaistaviin. Aineistona he käyttivät kuutta Eclipse-, Mozilla- ja
Gnome-säätiöiden avoimen lähdekoodin projektia. Heidän päätöspuihin perustuva
mallinsa luokitteli bugit 10-20\% sattumanvaraista arvausta paremmin.
Tutkimuksessa havaittiin ennusteen kannalta keskeisiksi muuttujiksi esimerkiksi
delegoitu kehittäjä ja avaamisen ajankohta. Avaamisen jälkeen muuttuvien
dynaamisten muuttujien mukaan ottaminen paransi mallin tarkkuutta. Mallin
heikkoutena voisi pitää sitä että opetus- ja testidataa ei oltu eroteltu
ajallisesti, eli mallissa oli tietoa "tulevaisuudesta" suhteessa testidataan.
Mallin käyttökelpoisuus uusien bugien korjausajan ennustamiseen ei siis ole
ollenkaan varmaa.

\textcite{Bhattacharya-2011} tarkastelivat eräitä muuttujia, joita on käytetty
bugikorjauksen valmistumisaikaa ennustavissa malleissa, ja huomasivat, että
pieni määrä muuttujia ei riitä selittämään ilmiötä. He myös huomasivat
ristiriidan aiempaan tutkimukseen \parencite{Guo-2010}, jossa todettiin
korrelaatio bugin raportoijan ja sen ratkeamisen välillä. Avoimen lähdekoodin
projekteissa tätä yhteyttä ei voitu todeta \parencite{Bhattacharya-2011}.

\textcite{Marks-2011} käyttivät Random Forest -algoritmia luodakseen tutkiakseen
aikaa bugin delegoinnista sen korjaamiseen. Tutkimuksen aineistona oli Eclipse-
ja Mozilla-projektien Bugzilla-tietokannat. Saadut mallit ennustivat korjausajan
n. 65\% tarkkuudella kolmeen luokkaan: kvartaalin, vuoden ja kolmen vuoden
sisällä korjattaviin. Tutkijat myös huomasivat, että parhaiten korjausaikaa
ennustavat muuttujat erosivat projektien välillä. Tämä antaa aihetta epäillä
sitä, kuinka tarkkaa mallia voi luoda, jos yleisesti hyvin toimivia muuttujia ei
löydy.

\textcite{Zhang-2013} tutkivat yhden ohjelmistoalan organisaation sisäisiä
projekteja, ja loivat kolme mallia estimoimaan eri näkökulmia bugien
korjausaikaan. Markovin ketjuihin perustuvalla mallilla he onnistuivat
luotettavasti ennustamaan tulevaisuudessa korjattavien ohjelmistovirheiden
lukumäärän. Monte Carlo -simulointiin perustuvalla metodilla he estimoivat
aikaa, joka kuluu annetun bugimäärän korjaamiseen. Luokittelemalla bugit
kNN-algoritmilla nopeasti ja hitaasti korjautuviin he myös onnistuivat
luokittelemaan yksittäisten bugien korjausaikoja (F-arvo 72.45\%). Tärkeänä
osana hyvän tuloksen saamista oli se, että tutkijat sovittivat mallinsa ottamaan
huomioon korjausaikojen jakauman, jossa suurin osa bugeista korjataan varsin
lyhyessä ajassa.

\textcite{Pfahl-2016} toistivat usean aiemmin bugien tai tehtävätikettiin
kuluvan ajan ennustamiseen esitetyn koneoppimismenetelmän, ja vertasivat saatua
tarkkuutta samojen tikettien luomisessa annettuun asiantuntija-arvioon.
Pääsääntöisesti asiantuntija-arviot olivat tarkempia, mutta Random Forest
-algoritmiin ja logistiseen regressioon perustuvat mallit pääsivät varsin
lähelle. Tekstianalyysiin perustuva \textit{Spherical k-means Clustering} tuotti
jopa tarkempia arvioita kuin asiantunijat.

\textcite{Porru-2016} kehittivät käytännön ohjelmistokehitykseen sopivan mallin,
joka ennustaa tehtävälle annettavien \textit{story point} -pisteiden määrän.
Niillä kuvataan tehtävän vaativuutta, ei niinkään arvioitua työaikaa. Paras
luokitinmalli saatiin aikaan tukivektorikoneella. Keskeisenä elementtinä
piirteiden louhinnassa oli tekstianalyysi, mihin tukivektorikoneet sopivat
hyvin. Mallin opetukseen riitti noin 300 tehtävätikettiä. Tehtävät jakautuivat
\textit{story point} -pisteiden mukaan 13:n luokkaan ja mallin keskimääräinen
tarkkuus oli 64 \%. Tutkimuksen aineistona oli kahdeksan avoimen lähdekoodin
projektia ja yksi teollisuusprojekti.

Myöhemmin \textcite{Scott-2018} tulivat kuitenkin siihen tulokseen, että
\textit{story point} -pisteitä ennustettaessa kehittäjäkohtaisten ominaisuuksien
perusteella kehitetty tukivektorikone-pohjainen malli tuottaa parempia tuloksia
kuin tekstianalyysiin pohjaava. \textcite{Scott-2018} käyttivät samaa aineistoa,
kuin \textcite{Porru-2016}, mutta heidän tekstianalyysiin pohjaavan mallin
tarkkuus oli vain 35.8\%, mikä on ristiriidassa aiemman tutkimuksen tulosten
kanssa. Koska myös kehittäjäkohtaisten ominaisuuksien vaikutuksesta tehtävien
valmistumisaikaan on ristiriitaisia tuloksia, vahvistavat nämä tutkimukset sitä
näkemystä, että myöskään bugin korjausajan ennustamista ei voida yksinkertaistaa
vain muutamaan muuttujaan.


\textcite{Ardimento-2017} käyttivät myös tukivektorikoneita mallissa, joka
ennustaa bugeja nopeasti tai hitaasti korjattavaiksi sen tiedon perusteella,
jota on käytettävissä tiketin avaushetkellä. Aineisona käytettiin kolmea avoimen
lähdekoodin projektia. Malli luokitteli bugitiketit nopeiksi 35-52\%
tarkkuudella, ja sen herkkyys (\textit{recall}) oli 60-76\%. 

\textcite{Ardimento-2016} lähestyivät samaa ennustustehtävää hyödyntämällä
SLDA-tekstinlouhinta-algoritmia. SLDA-algoritmin avulla he pyrkivät erottelemaan
jokaisen bugiraportin kuvauksesta useampia ala-aiheita, jotka voisivat selittää
paremmin bugien korjausaikaa. He painottivat malliaan tunnistamaan erityisesti
hitaasti korjattavat bugit. Aineistona heillä oli avoimen lähdekoodin Eclipse-,
Gentoo-, KDE- ja OpenOffice-projektit. Painotuksen vuoksi mallin tarkkuus jäi
alle tukivektorikoneen, mutta se tunnisti useimmat hitaasti korjattavat bugit.

\textcite{Al-Zubaidi-2017} käyttivät evoluutioalgoritmia tehtävän ratkaisemiseen
kuluvaa aikaa ennustavan mallin kehittämiseksi. Tavoitteena tehdä ennusteita
sillä tiedolla, jota on käytettävissä tiketin avaushetkellä. Tutkimuksen
lähdeaineistona oli viisi Apache-järjestön avoimen lähdekoodin projektia.
Kyseisellä aineistolla evoluutioalgoritmi tuotti paremman mallin kuin
lineaariseen regressioon, kNN-algoritmi tai Random Forest -algoritmi. Mallin
etuna useisiin muihin voinee pitää sitä, että luokitteluasteikollisten
aikaikkunoiden sijaan ennusteet ovat jatkuva-asteikollisia.

\parencite[][Esim.]{Lamkanfi-2012,ardimento-2020,lee-2020}
%,Ardimento-2020,Lee-2020, ramarao-2016}

\chapter{Tutkimusaihe/tutkimuskysymys}
Tutkimuksen tavoitteena on tuottaa lisää tietoa siitä, kuinka bugi-raportin
ratkeamisaikaa voidaan ennustaa käyttäen hyväksi tiedon louhimisen menetelmiä.
Pyrin käyttämään hyväksi aiemmassa tutkimuksessa \parencite{riivo-2016} melko
hyväksi todettua mallia ja dataa, ja testaamaan voiko mallia parantaa ottamalla
huomioon muissa tutkimuksissa tehtyjä havaintoja. Tutkimuksen tulosten avulla
voi tehdä arvioita siitä, kuinka kypsä tutkimussuunta on käytännön sovellusten
kannalta, ja mihin suuntaan huomio kannattaisi kiinnittää mallien kehittämisen
kannalta.

\chapter{Tutkimusstrategia/metodi ja sen valintaperusteet}
Tutkimusprosessin alussa on tarkoitus perehtyä syvällisesti käytettyyn malliin
\parencite{riivo-2016}, jotta tutkijalle syntyy sen tarjoamista
mahdollisuuksista ja rajoista. Tärkeää on tietää esimerkiksi, onko lähtödatan
käsittelyprosessi toistettavissa, ja millaisilla muuttujilla mallia on
mahdollista laajentaa. Sen lisäksi alussa on myös kerättävä mahdollisimman hyvin
kaikki aihetta käsittelevät tutkimukset. Kerättyjen tutkimusten perusteella
voidaan tehdä perusteltu valinta siitä, mitkä ovat potentiaalisimpia
vaihtoehtoja mallin kehittämiseksi.

Tämän jälkeen valittu/valitut muutokset toteutetaan ohjelmakoodiin ja
käytettävyyn dataan, sekä verrataan saadun uuden mallin tuloksia aiempiin
tutkimuksiin. Saadut tulokset ovat luonteeltaan kvantitatiivisia. 

Tutkimusen kokeellinen osuus on siis luonteeltaan tiedonlouhimis- ja
koneoppimistehtävä.

% \chapter{Aineiston keruun suunnittelu}
% - mitä, keneltä, milloin, miten
% - ml. eettiset näkökohdat (hyvä tieteellinen käytäntö jota noudatetaan,
%   tietosuoja, tieteellisen tutkimuksen rekisteriseloste)
\chapter{Aineisto}
Lähtökohtana aineistolle on aiemman tutkimuksen \parencite{riivo-2016}
oheismateriaalina julkaistu data ja lähdekoodi. Alkuperäinen data on saatu
GHTorrent-projektin \parencite{Gousi13} keräämän datasetin pohjalta.
GHTorrent-projekti kerää julkisesti saatavilla olevaa dataa Github-palvelusta
löytyvistä projekteista. %Lähteitä?
Projektiin kerättyä dataa on käytetty aiemmin useissa
tutkimuksissa. Datasta on poistettu käyttäjänimet ja sähköpostiosoitteet
tietosuoja-asetusten johdosta.

Jos tarpeellista ja mahdollista, voi olemassa olevaa aineistoa laajentaa
käyttämällä hyväksi GHTorrentin tarjoamaa dataa. Alkuperäisessä aineistossa on
dataa noin kahden vuoden ajalta, kun nykyään projektissa on dataa kuuden vuoden
ajalta.

ALkuperäiseen lähdekoodiin ei ole liitetty mitään tekijänoikeuslisenssiä, joten
lupa koodin uudelleenkäyttöön ja muokkaamiseen kysyttiin henkilökohtaisesti
sähköpostitse.
% - kuvaa konkreetilla tasolla miten ja milloin aineisto kerätään, käsitellään,
%   talletetaan, arkistoidaan/hävitetään
% - Aikataulu! \chapter{Aineiston analyysi} 
% - analyysin kuvaus, millä menetelmällä analyysi tehdään

%\chapter{Tulokset} 
%
%\chapter{Johtopäätökset}
\chapter{Aikataulu}
\begin{itemize}
  \item Maaliskuu: Teoreettisen viitekehyksen vahvistaminen, kirjallisuuteen
  perehtyminen.
  \item Huhtikuu:  Kirjallisuuteen perehtyminen, aineistoon perehtyminen.
  \item Toukokuu:  Aineistoon perehtyminen, kirjallisuuskatsauksen
  kirjoittaminen.
  \item Kesäkuu:   Ohjelmointi, datan käsittely.
  \item Heinäkuu:  Tulosten kerääminen. Tehdyn työn kirjaaminen.
  \item Elokuu:    Tulosten analysointi.
  \item Syyskuu:   Tulosten kirjaaminen.
  \item Lokakuu:   Tutkielman viimeistely
  \item Marraskuu: Palautus.
\end{itemize}

\printbibliography
\end{document}
